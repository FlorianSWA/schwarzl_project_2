\chapter{Einleitung}
Das Internet als globales Netzwerk ist schon seit geraumer Zeit ein zentrales Element unserer Gesellschaft. Die Technologien, auf denen es Aufgebaut ist, sind mannigfaltig und werden zunehmend mehr. Trotzdem sind alte und gut bewährte Ansätze weiterhin im Einsatz. Zu diesen bereits lange im Einsatz befindlichen Technologien gehört auch das Routing. Routing, zu Deutsch Weiterleitung, ist ein Kernkonzept einer jeden Kommunikation. Ohne eine funktionierende Weiterleitung von einem Knoten zum Nächsten ist eine Kommunikation über mehrere Stellen nicht möglich. Die Notwendigkeit für Routing besteht in jedem Netzwerk, welches aus mehreren Knoten besteht. Dieses Projekt befasst sich mit einem dieser Routing-Algorithmen, dem Distance-Vektor-Algorithmus. Das Projekt ist in C++ geschrieben und verwendet das Meson-Buildsystem.\footcite{meson}

\chapter{Aufgabenstellung}
Die Aufgabenstellung bestand daraus, den Routing-Algorithmus in einer lokalen Simulation zu implementieren. Die lokale Implementierung soll auf einer Maschine laufen, jedoch trotzdem den Netzwerk-Stack durchlaufen. Daher werden anstatt von IP- oder MAC-Adressen, welche normalerweise für das Routing verwendet werden, die Port-Nummern zur Identifikation verwendet. Zur Simulation des Algorithmus gehört auch ein fiktiver Netzwerkgraph. Jeder Knoten soll selbstständig sein und von anderen unabhängig seine Aufgaben erfüllen. Zum Aufbau der Netzwerkverbindung wird die Bibliothek asio verwendet.\footcite{asio} Asio ist eine weit verbreitete Bibliothek zur synchronen und asynchronen Netzwerkkommunikation in C++.